\chapter{General principles}

\section{Kinematics}\marginnote{This somehow creates  playing field}

\subsection{Deformations}

Let \(\Omega\subset\R^n\) be a \dhighlight{domain} \marginnote{Open, connected set}.

\begin{definition}\label{def:1.1}
    A \(C^k\) (\(k\geq 1\)) deformation of a domain \(\Omega\subset\R^n\) is a map \(\varphi:\Omega\to\R^n\) with:
    \begin{enumerate}
        \item \(\varphi\in C^k(\Omega,\R^n)\) 
        \item \(\varphi\) has a continuous extension to \(\overline{\Omega}\) and the extension is invertible
        \item \(\varphi\) is \dhighlight{orientation preserving}: \(\det D\varphi>0\) in \(\Omega\).
    \end{enumerate}
    \(\Omega\) is often called the \dhighlight{reference configuration} of the body and \(\varphi(\Omega)\) is often called 
    the \dhighlight{deformed configuration} (position in physical space).
\end{definition}

\subsubsection*{Some linear algebra}%TODO

Simplest example: \(\varphi(x)=Ax+b\) with \(b\in \R^n, A:\R^n\to\R^n\) linear with \(\det A>0\).
We often identify linear maps with the corresponding matrix with respect to the standard basis of \(\R^n\).
\[Ae_j=\sum_{i=1}^nA_{ij}e_i\]
We also usually use the standard scalar product on \(\R^n\)
\[x\cdot y=\sum_{i=1}^n x_iy_i\]

Transpose is defined by 
\[A^\intercal x\cdot y=x\cdot A y\]
for all \(x,y\in\R^n\).

\(A\) is symmetric if \(A^\intercal =A\). A symmetric matrix \(A\) is positive semi-definite if 
\[Ax\cdot x\geq 0\]
and positive definite if 
\[Ax\cdot x> 0.\]

\begin{definition}\label{def:1.2}
    \[O(n)=\{A:A^\intercal A=\Id\}\]
    \[\SO(n)=\{A:A^\intercal A=\Id, \det A=1\}\]
\end{definition}

Some basic facts: 
\begin{itemize}
    \item \(A\) isometry: \(|Ax|^2=|x|^2\forall x\in \R^n\) which holds \(\iff A\in O(n)\)
    \item \(A\in O(n)\implies \det A = \pm 1\)
\end{itemize}

Norms on matrices:
\begin{itemize}
    \item \dhighlight{Euclidean norm (Hilbert Schmidt norm)}: \(|A|^2=\text{tr}(A^\intercal A)=\sum_{i,j=1}^n A_{ij}^2, A\cdot B=\text{tr}A^\intercal B\)
    \item \dhighlight{Operator norm} \(\Vert A \Vert = \supp_{x\neq 0} \frac{|Ax|}{|x|}=\sup_{|x|=1}|Ax|\implies \Vert AB\Vert \leq \Vert A\Vert\Vert B\Vert \)
\end{itemize}

\begin{lemma}[Polar decomposition]\label{lem:1.3}
    Let \(A:\R^n\to\R^n\) linear, \(\det A>0\).
    \begin{enumerate}
        \item There exists a unique pair \((R,U)\) with \(R\in \SO(n), U\) symmetric s.t. \[A=RU\]
        \item there exists a unique pair \((R',V)\) with \(R'\in \SO(n), U\) symmetric s.t. \[A=VR'\]
    \end{enumerate}
    If \(\det A=0\) this is still possible, but we lose uniqueness.
\end{lemma}

\begin{theorem}\label{thm:1.4}
    Let \(W\) be symmetric and positive semi-definite. 
    Then there exists a unique symmetric and positive semi-definite matrix \(U\), such that \[W=U^2.\] 
\end{theorem}
\dhighlight{Notation:} \(U=\sqrt{W},U=W^{\frac{1}{2}}\).

\begin{proof}
    Just existence(not uniqueness).

    Let \(D=\text{diag}(d_i),d_i\geq 0\), then \(D^{\frac{1}{2}}=\text{diag}(\sqrt{d_i})\).
    For \(W\) symmetric positive definite, we get \[W=Q^\intercal D Q, Q\in \SO(n)\]
    Then \(U=Q^\intercal D^{\frac{1}{2}}Q\) is the square root: \[U^2=Q^\intercal D \underbrace{QQ^\intercal}_{=\Id}\] 
\end{proof}

\begin{proof}[Proof of \ref{lem:1.3}]
    For (ii) apply (i) to \(A^\intercal\).

    For (i): Assume we already knew that there are \((R,u)\) s.t. \[A=RU.\]
    Then \begin{align*}
        A^\intercal & = U^\intercal R^\intercal = UR^\intercal \\
        A^\intercal &A = U \underbrace{R^\intercal R}_{=\Id} U = U^2\\
        (A^\intercal A)^\intercal&=A^\intercal A \\
        (A^\intercal A x\cdot x &= (Ax\cdot A x)\geq 0)\\
        &\implies U=(A^\intercal A)^\frac{1}{2}
    \end{align*}

    If such a decomposition exists, \(U\) is unique by the formula.

    For existence define \(U=(A^\intercal A)^\frac{1}{2}\) and \(R\coloneqq AU^{-1}\). We have to 
    check that \(R\in \SO(n)\):
    \begin{align*}
        R^\intercal R &=(U^{-1})^\intercal A^\intercal A U^{-1}=\Id\\
        \det R &=\det A \det (U^{-1})>0\implies R\in \SO(n).
    \end{align*}
\end{proof}

\begin{lemma}\label{lem:1.5}
    \(A\in \R^{n\times n}\). Then there exists \(R,Q\in \SO(n)\) and \(\lambda_1\in \R,\lambda_2,\dots,\lambda_n\geq 0\) s.t. \(|\lambda_1|\leq \lambda_2\leq \dots \leq \lambda_n\):
    \[A=R\begin{bmatrix}
        \lambda_1 &&\\
        &\ddots & \\
        &&\lambda_n
    \end{bmatrix} Q.\]
    The \(\lambda_1,\dots,\lambda_n\) are uniquely determined by \(A\). The \(\lambda_i\) are called 
    the singular values of \(A\).
\end{lemma}

\begin{proof}
    If \(\det A>0\stackrel{\implies}{\text{Polar decomp.}} A=RU=\underbrace{R'Q^\intercal}_{R\in \SO(n)} D Q\).

    If \(\det A<0\) consider \(P=\begin{bmatrix}
        -1 & &&\\
        & 1 &&\\
        &&\ddots&\\
        &&&1
    \end{bmatrix}\implies \det(AP)>0\).

    The \(\vert A^\intercal A\vert\) are the eigenvalues of \((A^\intercal A)^\frac{1}{2}\).

\end{proof}

We know \(|A|=(\sum_{i=1}^n \lambda_i^2)^\frac{1}{2},\Vert A \Vert = \lambda_n\) and 
\[\det A=\prod_{i=1}^n \lambda_i\]

\subsubsection*{Rigid motion}


\begin{definition}\label{def:1.6}
    A deformation \(\varphi:\Omega\to \R^n\) is called a \dhighlight{rigid deformation} if \(D\varphi(x)\in \text{SO}(n)\forall x\in \Omega\).
\end{definition}

\begin{theorem}[Liouville]\label{thm:1.7}
    Suppose \(\varphi:\Omega\to\R^n\) is \(C^1\), \(\Omega\) is a domain and \(\det D\varphi\). Then the following three statements are equivalent:
    \begin{enumerate}
        \item \(D\varphi(x)\in\text{SO}(n)\forall x\in \Omega\) (Rigid motion)
        \item \(\varphi\) is an \dhighlight{affine} rigid motion: \[\varphi(x)=Ax+b, A\in \text{SO}(n),b\in \R^n\]
        \item \(|\varphi(x)-\varphi(y)|=|x.y|\forall x,y\in \R^n\)
    \end{enumerate} 
\end{theorem}

\begin{proof}
    Ideas (rest is homework):

    \begin{itemize}
        \item (ii) \(\implies\) (iii) is clear (since we assume the determinant to be $>0$)
        \item (iii) \(\implies\) (ii): nice ex. (true in hilbert spaces). \(\varphi((1-\lambda)x+\lambda y)=(1-\lambda)\varphi(x)+\lambda\varphi(y)\) By using that spheres are mapped to spheres (and two of those intersect in a single point)
        \item (ii) \(\implies\) (i): trivial
        \item (i) \(\implies\) (ii): goes via local version of (iii). Claim: for all \(x_0\in\Omega\exists r>0, B_r(x_0)\subset \Omega\) and \(|\varphi(x)-\varphi(y)|\leq |x-y|\forall x,y\in B_(0)\)
        \item \(\geq \)Use inverse function theorem 
    \end{itemize}
\end{proof}

This can be generalized to Sobolev spaces.

\markeol{01}